\documentclass[./report.tex]{subfiles}
\begin{document}

\subsection{Description of Algorithm}%

DAG (directed acyclic graph) is used to model task scheduling. He et al. published ``Bounding the Response Time of DAG Tasks Using Long Paths''(hereinafter referred to as \textit{"the paper"}) in 2022~\cite{heBoundingResponseTime2022}, which aims to provide a tighter bound of the Graham's Bound~\cite{grahamBoundsMultiprocessingTiming1969}. Graham's bound assumes all workloads other than the longest path delays the longest path. Thus, the bound is the sum of the longest path and the time taken to run other workloads ($len(G)$ is the length of longest path of $G$, $vol(G)$ is sum of all workloads of $G$, and $m$ is the number of processors):
\begin{equation}
   R\leq len(G)+\frac{vol(G)-len(G)}{m} 
\end{equation}
This bound is ``pessimistic'', because it considers all workloads not in the longest path to delay the longest path, while it is trivial to see that in many scenarios it is not true. The paper tighten the bound by identifying the workloads that can be executed parallel to the longest path.

To accomplish this, we need an algorithm to identify sets of vertices that satisfy:
\begin{itemize}
   \item all elements of a set executes sequentially, and
   \item all sets does not share vertices.
\end{itemize}

The paper identifies that for a set of vertices to satisfy the aforementioned requirement, they need not form a legitimate path. A "virtual path", which is a set of elements with a specified execution sequence can fulfill the requirement. This approach tightens the bound but depends on an unknown beforehand execution sequence. The concept of a \textit{Generalized Path} is proposed, which is a subset of vertices within a valid path in a DAG. It extends the notion of the longest path to this generalized path. By applying "workload swapping," which bases on the idea that certain swaps in the execution sequence do not affect the expected execution time, the paper proves the existence of a virtual path list that can be combined with a restricted critical path to form certain valid execution sequence, generalization of the special case. The paper further proves that a generalized path list containing the longest path is always bounded by some virtual path list, thus using such a generalized path list provides a practical bound on response time ($\lambda_i$ denotes the generalized path list):
\begin{equation}
   \label{eq:bound}
   R\leq\operatorname*{min}_{j\in[0,k]}\left\{l e n(G)+{\frac{v o l(G)-\sum_{i=0}^{j}l e n(\lambda_{i})}{m-j}}\right\}
\end{equation}
From the bound we can see that larger $len(\lambda_i)$ and smaller $k$ lead to tighter bounds, thus the list is generated by always finding the longest path remaining.

\subsection{Generate random schedule}%

To add randomization to schedule with least overhead:
\begin{itemize}
   \item As Eq.~\ref{eq:bound} shows, avoid the longest path, and add randomization to other workload, especially the other path in generalized path lists, since they are run parallel with the longest path while taking shorter time than it, leaving room for random waiting.
   \item Careful randomization might still be needed for longest path, as the longest path is likely to contain the beginning and ending of tasks' life circle, which is likely to be vulnerable, e.g. I/O. This can be done basing on the schedule of delaying.
   \item Develop real-time method of workload swapping. This is the method used for algorithm analysation in the paper, but it is not impossible to develop a practical swapping method.
\end{itemize}


\end{document}
